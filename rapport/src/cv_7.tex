\documentclass{article}
\usepackage[T1]{fontenc}
\usepackage[francais]{babel}
\usepackage[utf8]{inputenc}

\usepackage{amsmath,amsfonts,amsthm} % Math packages
\usepackage[pdftex]{graphicx}
\usepackage{hyperref}
\usepackage{lipsum}

\usepackage{listings}
\usepackage{charter}

% Listing Scala
\lstdefinelanguage{scala}{
  morekeywords={abstract,case,catch,class,def,%
    do,else,extends,false,final,finally,%
    for,if,implicit,import,match,mixin,%
    new,null,object,override,package,%
    private,protected,requires,return,sealed,%
    super,this,throw,trait,true,try,%
    type,val,var,while,with,yield},
  otherkeywords={=>,<-,<\%,<:,>:,\#,@},
  sensitive=true,
  morecomment=[l]{//},
  morecomment=[n]{/*}{*/},
  morestring=[b]'',
  morestring=[b]',
  morestring=[b]"''
}

\usepackage{color}
\definecolor{dkgreen}{rgb}{0,0.6,0}
\definecolor{gray}{rgb}{0.5,0.5,0.5}
\definecolor{mauve}{rgb}{0.58,0,0.82}

% Default settings for code listings
\lstset{frame=tb,
  language=scala,
  aboveskip=3mm,
  belowskip=3mm,
  showstringspaces=false,
  columns=flexible,
  basicstyle={\small\ttfamily},
  numbers=none,
  numberstyle=\tiny\color{gray},
  keywordstyle=\color{blue},
  commentstyle=\color{dkgreen},
  stringstyle=\color{mauve},
  frame=single,
  breaklines=true,
  breakatwhitespace=true
  tabsize=3
}

% 
\begin{document}

\title{Picobot}

\author{PHILIPPON BOSSUT BAILLEUL}


\maketitle
\newpage
\tableofcontents

\newpage


\section{Introduction}
Picobot est un programme utilisé comme support au cours d'informatique
du Dr. Zachary Dodds au Harvey Mudd College de Californie.
% ajouter lien
\\

Son but est de simuler le parcours d’un robot dans un environnement
donné en utilisant des règles dictant son comportement.
\\

Ce projet utilise le langage de spécification déclaratif Alloy pour
générer ces règles. L'intérêt est de chercher le plus petit ensemble
de règles qui complète le niveau, quelque soit la position initiale du
robot.



\section{Picobot}
\subsection{Environnement}
Un environnement est modélisé sous forme d’un tableau où le robot doit
parcourir au moins une fois sur chaque case grâce aux règles
pré-établies.
Comme le robot n’est pas doté de mémoire interne alors il ne peut pas
savoir où il se trouve. Il ne peut donc pas utiliser d’algorithmes de
recherche de chemin (Dijkstra, A*).
\\
\subsection{Règles}
Une règle est formée de la façon suivante:

\textit{CurrentState Surroundings -> MovementDirection NewState}\\

\begin{itemize}
\item {\textit{CurrentState} et \textit{NewState} sont deux nombres entiers (comme 0,1,2,
  etc).}
\item {\textit{MovementDirection} est soit N, E, W, S, ou X. «X» signifie ``ne pas
bouger''.\\}
\end{itemize}

Une fois ces quatre facteurs déterminés, la formule suivante est
calculée et son résultat est sauvegardée en
mémoire. Quand toutes les méthodes sont évaluées, l'algorithme renvoie
la liste des dix méthodes les plus suspectes.



\subsection{Exemple}

\section{Résultats}



\section{Conclusion}
\newpage
\section{Exemple Java}
\begin{lstlisting}
  package examples

  /** This examples illustrates the use of Guarded
  *  Algebraic Data Types in Scala. For more information
  *  please refer to the Scala specification available
  *  from http://scala-lang.org/docu/.
  */
  object gadts extends Application {

    /* The syntax tree of a toy language */
    abstract class Term[T]

    /* An integer literal */
    case class Lit(x: Int) extends Term[Int]

    /* Successor of a number */
    case class Succ(t: Term[Int]) extends Term[Int]

    /* Is 't' equal to zero? */
    case class IsZero(t: Term[Int]) extends Term[Boolean]

    /* An 'if' expression. */
    case class If[T](c: Term[Boolean],
    t1: Term[T],
    t2: Term[T]) extends Term[T]

    /** A type-safe eval function. The right-hand sides can
    *  make use of the fact that 'T' is a more precise type,
    *  constraint by the pattern type.
    */
    def eval[T](t: Term[T]): T = t match {
      case Lit(n)        => n

      // the right hand side makes use of the fact
      // that T = Int and so it can use '+'
      case Succ(u)       => eval(u) + 1
      case IsZero(u)     => eval(u) == 0
      case If(c, u1, u2) => eval(if (eval(c)) u1 else u2)
    }
    println(
    eval(If(IsZero(Lit(1)), Lit(41), Succ(Lit(41)))))
  }
\end{lstlisting}



\end{document}
